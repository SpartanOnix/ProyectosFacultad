\documentclass{article}
\usepackage[utf8]{inputenc}
\usepackage{ upgreek }
\usepackage{ amssymb }

\title{Lógica Computacional. Tarea 3}
\author{Sandoval Mendoza Antonio }
\date{March 2020}

\usepackage{natbib}
\usepackage{graphicx}

\begin{document}

\maketitle

\section{Realiza las siguientes sustituciones:}

(a) $(((\forall x (\exists yP^2_2(x,z))\vee P^1_2(y)))[x:=f^1_1(z)])[y:=f^2_3(z,x)]$\\
Aplicamos sustitucion de x a ambas formulas por separado\\
= $(\forall x (\exists yP^2_2(x,z)[x:=f^1_1(z)]) \vee P^1_2(y)[x:=f^1_1(z)])[y:=f^2_3(z,x)]$\\
No se pudo sustituir ninguna variable. Regla 3a de las laminas\\
= $(((\forall x (\exists yP^2_2(x,z)) \vee P^1_2(y))))[y:=f^2_3(z,x)]$\\
Aplicamos sustitucion de y a ambas formulas por separado\\
= $(\forall x (\exists yP^2_2(x,z)))[y:=f^2_3(z,x)]\vee P^1_2(y)[y:=f^2_3(z,x)]$\\
= $\forall x (\exists yP^2_2(x,z)[y:=f^2_3(z,x)])\vee P^1_2(y)[y:=f^2_3(z,x)]$\\
No se pudo sustituir ninguna variable del lado izq. Regla 3a\\
= $\forall x (\exists yP^2_2(x,z))\vee P^1_2(y)[y:=f^2_3(z,x)]$\\
= $\forall x (\exists yP^2_2(x,z))\vee P^1_2(y[y:=f^2_3(z,x)])$\\
Sustituimos y\\
= $\forall x (\exists yP^2_2(x,z))\vee P^1_2(f^2_3(z,x))$\\
\\
\\
(b) $(((\exists x(\forall z(\exists yP^3_1(x,y,z))))\Leftrightarrow P^3_2(x_1,y,z_2))[x_1:=f^3_1(x,y,z_3)])[z:=f^1_2(y)]$\\
Alfa-Equivalencia ya que "x" y "y" aparecen del lado derecho de la sustitucion\\
= $(((\exists a(\forall z(\exists bP^3_1(a,b,z)))) \Leftrightarrow P^3_2(x_1,b,z_2))[x_1:=f^3_1(x,y,z_3)])[z:=f^1_2(y)]$\\
Aplicamos sustitucion de $x_1$ a ambas formulas por separado\\
= $((\exists a(\forall z(\exists bP^3_1(a,b,z)[x_1:=f^3_1(x,y,z_3)])))\Leftrightarrow P^3_2(x_1,b,z_2)[x_1:=f^3_1(x,y,z_3)])[z:=f^1_2(y)]$\\
Solo sustituimos por la derecha \\
= $((\exists a(\forall z(\exists bP^3_1(a,b,z)[x_1:=f^3_1(x,y,z_3)])))\Leftrightarrow P^3_2(f^3_1(x,y,z_3),b,z_2))[z:=f^1_2(y)]$\\
NO podemos sustituimos por la izq. Regal 3a\\
= $(((\exists a(\forall z(\exists bP^3_1(a,b,z))))\Leftrightarrow P^3_2(f^3_1(x,y,z_3),b,z_2))[z:=f^1_2(y)]$\\
Aplicamos sustitucion de z a ambas formulas\\
= $((\exists a(\forall z(\exists bP^3_1(a,b,z)))))[z:=f^1_2(y)]\Leftrightarrow P^3_2(f^3_1(x,y,z_3),b,z_2)[z:=f^1_2(y)]$\\
Aplicamos la sustitucion más interna del lado izq\\
= $\exists a(\forall z(\exists bP^3_1(a,b,z)[z:=f^1_2(y)]))\Leftrightarrow P^3_2(f^3_1(x,y,z_3),b,z_2)[z:=f^1_2(y)]$\\
No podemos sustituir z del lado izq porque esta ligada y lado derecho regla 3a\\
= $\exists a(\forall z(\exists bP^3_1(a,b,z)))\Leftrightarrow P^3_2(f^3_1(x,y,z_3),b,z_2)$\\
\\
\\
(c) $((\exists xP^3_1(x,y,z)) \Rightarrow P^1_1(x))[z:=f^2_1(x,y)]$\\
Como "x" y "y" aparecen del lado derecho de la sust, usamos alfa-equiv\\
=$((\exists aP^3_1(a,y,z)) \Rightarrow P^1_1(a))[z:=f^2_1(x,y)]$\\
Ahora con la variable "y"\\
=$((\exists aP^3_1(a,b,z)) \Rightarrow P^1_1(a))[z:=f^2_1(x,y)]$\\
Aplicamos sustitucion de z a ambas formulas\\
=$(\exists aP^3_1(a,b,z)[z:=f^2_1(x,y)]) \Rightarrow P^1_1(a)[z:=f^2_1(x,y)]$\\
No sustituimos nada del lado derecho, regla 3a\\
=$(\exists aP^3_1(a,b,z)[z:=f^2_1(x,y)]) \Rightarrow P^1_1(a)$\\
No podemos sustituir z del lado izq porque esta ligada\\
=$((\exists aP^3_1(a,b,z)) \Rightarrow P^1_1(a))$\\
Quitamos parentesis\\
=$\exists aP^3_1(a,b,z) \Rightarrow P^1_1(a)$\\
\\
\\
(d) $(((\forall yP^3_1(x,y,z))\bigwedge (\exists zP^3_2(x,y,z))\bigwedge (\forall xP^3_3(x,y,z)))[x:=f^2_1(y,z)])[y:=f^2_2(x,z)]$\\
Aplicamos sustitucion de x a las formulas\\
= $((\forall yP^3_1(x,y,z)[x:=f^2_1(y,z)])\bigwedge (\exists zP^3_2(x,y,z)[x:=f^2_1(y,z)])\bigwedge (\forall xP^3_3(x,y,z)[x:=f^2_1(y,z)]))[y:=f^2_2(x,z)]$ \\
Alfa-Equivalencia "a" por "y" y "b" por "z"\\
= $((\forall aP^3_1(x,a,b)[x:=f^2_1(y,z)])\bigwedge (\exists bP^3_2(x,a,b)[x:=f^2_1(y,z)])\bigwedge (\forall xP^3_3(x,a,b)[x:=f^2_1(y,z)]))[y:=f^2_2(x,z)]$ \\
No podemos sustituir x porque es variable ligada\\
= $((\forall aP^3_1(x,a,b))\bigwedge (\exists bP^3_2(x,a,b))\bigwedge (\forall xP^3_3(x,a,b)))[y:=f^2_2(x,z)]$ \\
Aplicamos sustitucion de y a las formulas\\
= $((\forall aP^3_1(x,a,b)[y:=f^2_2(x,z)])\bigwedge (\exists bP^3_2(x,a,b)[y:=f^2_2(x,z)])\bigwedge (\forall xP^3_3(x,a,b)[y:=f^2_2(x,z)]))$ \\
Alfa-Equivalencia "c" por "x"\\
= $((\forall aP^3_1(c,a,b)[y:=f^2_2(x,z)])\bigwedge (\exists bP^3_2(c,a,b)[y:=f^2_2(x,z)])\bigwedge (\forall cP^3_3(c,a,b)[y:=f^2_2(x,z)]))$ \\
No podemos sustituir y \\
= $((\forall aP^3_1(c,a,b))\bigwedge (\exists bP^3_2(c,a,b))\bigwedge (\forall cP^3_3(c,a,b)))$ \\
Quitamos parentesis\\
= $\forall aP^3_1(c,a,b)\bigwedge \exists bP^3_2(c,a,b)\bigwedge \forall cP^3_3(c,a,b)$ \\


\maketitle
\section{ Encuentra un modelo para el siguiente conjunto de fórmulas: }
\\
$\forall x(\exists yP^2_1(y,x)),\forall x \neg P^2_2(x,c) \Rightarrow P^2_1(x,c), \neg \exists xP^2_2(c, f^1_1(x)), \neg \exists xP^2_2(x, f^1_1(x))$
\\
\\
\\
Empezamos con que todo es igual a $\Gamma$ entonces: 
\\
$\forall x(\exists yP^2_1(y,x)),\forall x \neg P^2_2(x,c) \Rightarrow P^2_1(x,c), \neg \exists xP^2_2(c, f^1_1(x)), \neg \exists xP^2_2(x, f^1_1(x)) = \Gamma$
\\
Separamos los predicados:\\ \\
1. $\forall x(\exists yP^2_1(y,x))$\\ \\
2. $\forall x \neg P^2_2(x,c) \Rightarrow P^2_1(x,c)$\\ \\
3. $\neg \exists xP^2_2(c, f^1_1(x))$\\ \\
4. $\neg \exists xP^2_2(x, f^1_1(x))$\\

Y tenemos que  $I_\mathds{Z} = < \Uppsi,\Phi , \sqcap >$ en especial un universo:\\ \\
$\Uppsi(x)=1$, $\Uppsi(y)=1$, $\Uppsi(c)=0$ 
\\ \\ $\sqcap(P^2_1)$= \{n = m\}
\\ $\sqcap(P^2_2)$= \{n $\geqslant$ m\}
\\ $\sqcap(f^1_1)$= \{n tal que n = (\mid n \mid + 1)$\} (valor abs)
\\ \\
Entonces se cumplen las 4
\\ \\
1- $\forall x(\exists yP^2_1(y,x))$\\
$\sqcap(P^2_1(0,0))$= \{0 = 0\}\\
SI SE CUMPLE 
\\ \\
2- $\forall x \neg P^2_2(x,c) \Rightarrow P^2_1(x,c)$\\
$\sqcap(\neg P^2_2(-1,0) \Rightarrow P^2_1(-1,0))$=\\
$\{-1 $\geqslant$ 0\} \Rightarrow \{-1 = 0\}$\\
Falso de ambos lados de la implicación por lo que da verdadero\\
SI SE CUMPLE 
\\ \\
3. $\neg \exists xP^2_2(c, f^1_1(x))$\\
$\sqcap(P^2_2(0, f^1_1(-3))$=\\
$\sqcap(P^2_2(0, \mid -3 \mid +1))$=\\
$\sqcap(P^2_2(0, 4))$= \{0 $\geqslant$ 4\}\\
Vemos que no existe una x que cumpla la condición\\
entonces SI SE CUMPLE 
\\ 
\\
4. $\neg \exists xP^2_2(x, f^1_1(x))$\\
\sqcap(P^2_2(9, f^1_1(-9))$=\\
$\sqcap(P^2_2(9, \mid -9 \mid +1))$=\\
$\sqcap(P^2_2(9, 10))$= \{9 $\geqslant$ 10\}\\
Vemos que no existe una x que cumpla la condición\\
entonces SI SE CUMPLE 
\\ 
\\
Los 4 predicados son Verdaderos por lo que al dar una Interpretación que para cada formula de gama se cumple, entonces concluimos con que la interpretación es un modelo para Gama.\\
Se cumple $\models_I \gamma$ y $\gamma \in \Gamma$

\maketitle
\section{ Demuestra los siguientes teoremas de deducción natural:
}

(a) $\forall x(\exists yP^1_1(x) \Rightarrow P^1_2(y)) \vdash_N \neg(\exists x (\forall yP^1_1(x) \bigwedge \neg P^1_2(y)));$
\\
\\
1. \hspace{3mm}$\forall x(\exists yP^1_1(x) \Rightarrow P^1_2(y))$ \hspace{2.5cm} Premisa\\ \\
2. \hspace{3mm}$\neg (\forall x(\exists yP^1_1(x) \Rightarrow P^1_2(y)))$ \hspace{2cm} por  I$\neg$ \\
3. \hspace{3mm} $(\exists x\neg (\exists yP^1_1(x) \Rightarrow P^1_2(y)))$ \hspace{1.9cm} por equivalencia de $\neg\forall$ \\
4. \hspace{3mm} $(\exists x(\forall y \neg (P^1_1(x) \Rightarrow P^1_2(y))))$ \hspace{1.6cm} por equivalencia de $\neg\forall$ \\
5. \hspace{3mm} $\exists x(\forall y $P^1_1(x) \bigwedge \neg P^1_2(y))$  \hspace{2.3 cm} por $\neg (p\Rightarrow q) \equiv p \bigwedge \neg q$ \\
\\
6. \hspace{3mm} $\neg (\exists x(\forall y $P^1_1(x) \bigwedge \neg P^1_2(y)))$  \hspace{1.8 cm} por I$\neg$ \\
\\
\\
\\
(b) $\forall x\forall yP^1_1(x) \Rightarrow P^1_2(x)\bigwedge P^1_3(y),P^1_1(c)  \vdash_N \neg \forall z \neg P^1_2(z)$
\\
\\
1. \hspace{3mm} $\forall x\forall yP^1_1(x) \Rightarrow P^1_2(x)\bigwedge P^1_3(y)$ \hspace{1.5cm} Premisa\\
2. \hspace{3mm} $P^1_1(c)$ \hspace{5.1cm} Premisa\\ \\
3. \hspace{3mm} $P^1_1(c) \Rightarrow P^1_2(c)\bigwedge P^1_3(v)$ \hspace{2.4cm} por E $\forall x,y$\\
4. \hspace{3mm} $P^1_2(c)\bigwedge P^1_3(v)$ \hspace{3.8cm} por E $\Rightarrow$ 3,2\\
5. \hspace{3mm} $P^1_2(c)$ \hspace{5.1cm} por E $\bigwedge$\\
6. \hspace{3mm} $P^1_3(v)$ \hspace{5.1cm} por E $\bigwedge$\\
\\
7. \hspace{3mm} $\neg(\forall z \neg P^1_2(z)) \equiv \exists z P^1_2(z)$\\
8.\hspace{3mm} $\exists z P^1_2(z)$ \hspace{4.8cm} por I$\exists$5 por equivalencia de $\neg\forall$\\
\\
\\
(c)$\neg(\exists x(P^1_1(x) \bigwedge (\forall y \neg P^2_1(x,y)))) \vdash_N \forall x(P^1_1(x) \Rightarrow (\exists y P^2_1(x, y)))$
\\
\\
1. \hspace{3mm} $\neg(\exists x(P^1_1(x) \bigwedge (\forall y \neg P^2_1(x,y))))$ \hspace{1.5cm} Premisa
\\ 
2. \hspace{3mm} $\forall x\neg (P^1_1(x) \bigwedge (\forall y \neg P^2_1(x,y)))$ \hspace{1.8cm} por equivalencia de $\neg\forall$\\ \\
3. \hspace{3mm} $\forall x\neg P^1_1(x) \vee \neg (\forall y \neg P^2_1(x,y))$ \hspace{1.8cm} distributividad de la negación\\
4. \hspace{3mm} $\forall x\neg P^1_1(x) \vee \exists y \neg(\neg P^2_1(x,y))$ \hspace{1.8cm} por equivalencia de $\neg\forall$ \\ \\
5. \hspace{3mm} $\forall x\neg P^1_1(x) \vee \exists y P^2_1(x,y)$ \hspace{2.5cm} eliminamos doble negación \\
6. \hspace{3mm} $\forall x(P^1_1(x) \Rightarrow (\exists y P^2_1(x,y)))$ \hspace{2cm} por p \Rightarrow q \equiv \neg p\vee q \\





\end{document}



















