\documentclass{article}
\usepackage[utf8]{inputenc}
\usepackage[spanish]{babel} 
\usepackage[table]{xcolor}
\usepackage{amssymb}
\usepackage{authblk}
\usepackage{enumitem}
\usepackage{float}
\usepackage{forest}

\title{Practica 03}
\author{Dante Jusepee Sinencio Granados}
\affil{Facultad de ciencias, UNAM}
\date{April 2020}

\begin{document}

\maketitle

\section{Introduction}
En esta practica creamos decodificadores (y un codificador) con compuertas logicas.

\section{Preguntas}
\subsection{Decodificador 3x8}
Para este lo primero fue crear el 2x4 y el 2x4 enable, despues lo logico es combinar 2 de 2x4 pero se obtendria un 4x8 en lugar de 3x8, ahi es donde entra el 2x4 enable, con este ya es mas facil unirlos, las 2 primeras entradas son para obtener los valores y la tercera para cambiar el valor del enable.

\subsection{Problema de Toby}
Regresaremos cuando inicio el club de toby, en esa entonces solo era toby y su mejor amigo, 2 amigos mas se quieren unir, pero para unirse, toby (tienes 2 votos por norma al crear el club) y su mejor amigo deben de votar a favor, si uno de los 2 se niega, entonces no se pueden unir, el unico inconveniente es que toby ya les habia dicho que si se podian unir (entonces uno de los votos de toby siempre estara a favor).
\\
Con el decodificador de 2x4 enable se puede resolver ya que el enable es una entrada que puedes tener como quieras y las otras son las que varias y con compuertas logicas se puede resolver con 2 AND uno donde los 2 votos de toby coincidan y otro donde el voto del mejor amigo y los 2 de toby coincidan.

\subsection{Decodificador 4x16}
Para este se nesecitan un decodificador de 2x4 y cuatro decodificadores de 2x4 enable, el 2x4 sirve para dar valor a las entradas enable de los 2x4 enable, 2 entradas sirven para los valores de los 2x4 enable y las otras 2 entradas para el 2x4.

\section{Punto extra}
\subsection{Decodificador 5x32}
Esta en el archivo de logisim (esta muy grande este circuito).

\subsection{Codificador 16x4}
Esta en el archivo de logisim, Para estre lo cree en vase a codificadores 4x2 y 8x3, todo lo cree yo ya que casi no encontre ejemplos de como crear un codificador con compuertas logicas.

\subsection{Series o peliculas}
No se que servicios tengas pero lo dividire en series, peliculas, caricaturas y anime.
\subsubsection{Series}
- Batman 1969
- The ofice
- Suits
- The boys
- Good omen

\subsubsection{Peliculas}
- Hardcore henry\\
- Kung fury\\
- Train to busan\\
- Red line (Pelicula en anime

\subsubsection{Caricaturas}
- Final space\\
- Invader zim\\
- Todo ben 10\\
- Sonic boom\\
- Teen titans go\\
- Avatar (incluyendo la leyenda de korra)

\subsubsection{Anime}
- Konosuba\\
- Haruhi susumiya\\
- Bunny girl senpai\\
- No game no life\\
- Megalobox\\
- Jojo's\\
- Pokemon generation\\
- Mob psyco 100\\
- One punch man\\
- Kakegurui\\
- Terror in resonant\\
- Goblin slayer\\
- Angel bits


\end{document}
