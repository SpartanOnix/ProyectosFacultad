\documentclass{article}

\usepackage[spanish]{babel}
\usepackage[T1]{fontenc}
\usepackage{lmodern}
\usepackage{mathtools}
\usepackage{selinput}
\usepackage{enumerate}
% \usepackage{tikz}
\usepackage{amssymb}
%Ajustando los margenes
\usepackage[
top=3cm,
bottom=3cm,
left=2cm,
right=2cm,
heightrounded,
]{geometry}
%Configurando los acentos
\SelectInputMappings{
aacute={á},
ntilde={ñ}
}
\title{Bibliografia}
\begin{document}
  \maketitle

  En este proyecto nos basamos en su totalidad de la documentacion de Vala, ya que como es un lenguaje experimental
  casi no hay informacion de este, entre las distintas paginas que encontramos las que mas frecuentamos fueron:

  \begin{enumerate}
    \item https://wiki.gnome.org/Projects/Vala/Documentation

    \item https://valadoc.org/

    \item http://www.devjoker.com/contenidos/catss/140/Importacion-de-espacios-de-nombres.aspx
  \end{enumerate}
\end{document}
